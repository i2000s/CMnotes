\chapter{Singular perturbation theory}\label{chap:singularperturbation}
\section{Solving a high-order differential equation using perturbation method}
\textbf{Sep 13}.

Equations with nonlinear terms, for example, 
\begin{equation}
\sdt{x}+ \alpha \dt{x} + \omega^2 x = 0
\end{equation}
can be modified as
\begin{equation}
\sdt{x}+ \alpha \left(\dt{x}\right)^{17} + \omega^2 x = 0.
\end{equation}
Can we solve it analytically? This is the technique to learn in this section. One technique to be introduced is called singular perturbation method.

Suppose we have an equation of 
\begin{equation}
\frac{d^2y}{dT^2}+ A \left(\frac{dy}{dT}\right)^{2n+1} + \omega^2 y = 0.
\end{equation}
PS: one can introduce a quantity of $ x=y/y_0 $ and/or a quantity of $ t=T/T_0 $ to make a differential equation to be dimensionless so that one do not need to worry about the dimensions. One dimensionless equation can be 
\begin{equation}
\frac{d^2x}{dt^2}+ \alpha \left(\frac{dx}{dt}\right)^{2n+1} + \omega^2 x = 0,\quad n=0,\,1,\,\cdots 
\end{equation}

Suppose the initial condition gives
\begin{equation}
x(0)=0,\quad \dot{x}(0)=B.
\end{equation}
Now, if $ \alpha=0 $, $ x(t)=B\sin t $. If $ \alpha\neq 0 $, we can set $ x=x_0 +\alpha x_1+ \alpha^2 x_2 + \cdots $ using the normal perturbation method. 

\textit{e.g.} $n=1$, so that 
\begin{equation}
\sdt{x} + \alpha \left( \dt{x}\right)^3 + x =0.\label {alphasdt}
\end{equation}
Using perturbation method, we have
\begin{equation}
\frac{d^2}{dt^2}(x_0 + \alpha x_1 + \cdots)+ \alpha \left[ \dt{x_0} + \alpha \dt{x_1} + \cdots \right]^3 +(x_0 + \alpha x_1 + \cdots)=0.
\end{equation}
\begin{align}
\sdt{x_0} + x_0 &=0, \label{pert1}\\
\sdt{x_1} + \left( \dt{x_0}\right)^3 + x_1 &= 0 \label {pert2},
\end{align}
which are the outcome of the perturbation theory we usually use. 

Equ.~\eqref{pert1} gives
\begin{equation}
x_0(t) = B\sin t.
\end{equation}
Equ.~\eqref{pert2} gives
\begin{equation}
\left(\dt{x_0}\right)^3 = B^3 cos^3 t= \frac{B^3}{4}(cos 3t + 3 cos t),
\end{equation}
using  $ \cos^3 t= \frac{cos 3t + 3 cos t}{4}$. 

Now, Equ.~\eqref{pert2} gives
\begin{equation}
\sdt{x_1}+ x_1 = -\frac{B^3}{4} \cdot \underbrace{\left[\cos 3t + 3 \cos t\right] }_{\text{one fast oscillating term $ + $ an important term}}.
\end{equation}
This equation gives an oscillation that it oscillates normally at the beginning, but will blow off consequently. The terms in the brace are called secular terms. 



It can be shown that the normal perturbation method fails for any $ n $. 

\section{Singular perturbation method}
\textbf{Sep 18}.

A better method: singular perturbation method. This method is based on treating the equations above as partial differential equations with two time scales, which include a fast time scale $ \tau $ and a slow time scale $ t $. With these two time scales, one can convert an ODE $ (t) $ to a PDE $ (t,\tau) $. 

An example,
\begin{equation}
\sdt{x}+ \beta \dt{x}+\omega^2 x=0
\end{equation}
gives
\begin{align}
\epsilon^2 \tilde{x} -\epsilon x_0 - \dot{x_0}+\beta (\epsilon \tilde{x}-x_0)+ \omega^2 \tilde{x}&=0,\\
(\epsilon^2+ \beta \epsilon + \omega^2)\tilde{x}&=\dot{x_0}+ (\epsilon + \beta)x_0
\end{align}
\begin{align}
\Rightarrow \tilde {x}= \frac{\dot{x_0}}{\epsilon^2+\beta \epsilon + \omega^2}
\end{align}

\begin{align}
\tilde{x}= \frac{B}{(\epsilon +\beta/2)^2 + (\omega^2-\beta^2/4)}=\frac{B}{\Omega}\left( \frac{\Omega}{(\epsilon+ \beta/2)^2+\Omega^2}\right),
\end{align}
where $ \Omega = \sqrt{\omega^2 - \frac{\beta^2}{4}} $.  $ \Omega \approx \omega $ in most cases, and 
\begin{align}
x(t)= \frac{B}{\Omega}e^{-\frac{\beta t}{2}}\sin \Omega t.\label {xtdamping} 
\end{align}
This solution has the main feature that $ x(t)=B \frac{\sin t}{\sqrt{\cdots t}} $. The result can be plotted in the figure below. 
\scalefig{../Figs/handdraw918_1}{0.5}{Diagram a damping oscillation described in Equ.~\eqref{xtdamping}.}
%\missingfigure{918-1.}

Let make $ \tau =\delta t = \epsilon t $, where $ \epsilon  $ is small. We go back to solve Equ.~\eqref{alphasdt}. Now, 
\begin{align}
\dt{x } = \frac{\partial x}{\partial t}+ \frac{\partial x}{\partial \tau}\frac{d\tau}{dt}= \frac{\partial x}{\partial t}+ \epsilon \frac{\partial x}{\partial \tau}\label {singularp1}
\end{align}
\begin{align}
\sdt{x} &= \frac{\partial ^2 x}{\partial t^2}+ \epsilon \frac{\partial ^2 x}{\partial t \partial \tau} + \epsilon \left[ \frac{\partial ^2 x}{\partial t \partial \tau}+ \epsilon \frac{\partial ^2 x}{\partial \tau ^2}\right] \\
%\end{align}
%\begin{align}
&=\frac{\partial ^2 x}{\partial t^2} + 2\epsilon \frac{\partial ^2 x}{\partial t \partial \tau}+ \epsilon^2  \frac{\partial ^2 x}{\partial \tau^2}.\label{singularp2}
\end{align}
We can substitute the equations above to Equ.~\eqref{alphasdt} to give
\begin{align}
\frac{\partial ^2 x}{\partial t^2}+ \ldots 
\end{align}
We can regard $\epsilon \sim \alpha $ (in the same order). 

Now let us expand $ x $ in terms of $ \epsilon $ as 
\begin{align}
x=x_0 + \epsilon x_1 + \epsilon^2 x_2^2 + \ldots .
\end{align}
The partial differential equation will give 
\begin{align}
\frac{\partial ^2 x_0 }{\partial t^2}+ x_0 &= 0,\label {partialpert1}
\end{align}
but you will not get $ x_0(t)=B\sin t $, rather $ x_0 (t)= B(\tau) \sin t $. 

PS: Finally, one should obtain $ x_0(t)=\frac{\sin t}{\sqrt{\cdots t}} $. The zero order solution can characterize the main feature of the exact solution of the equation!

The first order perturbation equation is
\begin{align}
\frac{\partial ^2 x_1}{\partial t^2} +x_1 = -2 \frac{\partial ^2 x_0}{\partial t \partial \tau} - \frac{\alpha}{\epsilon} \left( \frac{\partial x_0}{\partial t}\right) . \label {partialpert2}
\end{align}
...

Return to the linear problem which is 
\begin{align}
\sdt{x} + \alpha \dt{x} + \omega^2 x=0.\label {sdewithomega}
\end{align}
Divided by $ \alpha $ to give
\begin{align}
\left( \dfrac{1}{\alpha}\right)\sdt{x} + \dt {x} + \frac{\omega^2}{\alpha}x &=0 \\
\Rightarrow \dt{x} + \Gamma x &= -\frac{1}{\alpha } \sdt{x}.
\end{align}
In some sense $ \alpha \rightarrow \infty $ (Arrestotal approximation), the equation above gives
\begin{align}
\dt{x} +\Gamma x=0.  \label {Gammaapprox}
\end{align}

The solution gives a damping oscillation with two terms as shown in the figure below.
\scalefig{../Figs/handdraw918_2}{0.95}{Solution of a damping oscillation.}
%\missingfigure{918-2.}
If one term is ignored, it will become a pure damping movement. 

   
\textbf{Sep 20}

A summary of singular perturbation theory: For Equ.~\eqref{alphasdt}, using normal perturbation theory, one has
\begin{equation}
x(t)= x_0(t)+ \alpha x_1(t) +\alpha^2 x_2(t).
\end{equation}

\begin{equation}
\rightarrow \left\{ \begin{array}{c}
\sdt{x_0} + x_0 =0\Rightarrow x_0=B\sin t\\
\sdt{x_1}+ x_1 =\text{dangerous terms}
\end{array}\right.
\end{equation}
There is a dangerous term. Let us see how to avoid this dangerous term in the improved singular perturbation theory. 

Using the singular perturbation theory, we make $ x=x(\tau=\epsilon t,t) $ with perturbation coefficient $ \epsilon $ and/or $ \alpha $. 
The coupled partial differential Equs.~\eqref{singularp1} and~\eqref{singularp2} lead to 
Equs.~\eqref{partialpert1} and~\eqref{partialpert2}. 

Back to the result:
\begin{align}
\frac{\partial^2 x_1}{\partial t^2}+x_1 = -2 \frac{dB(\tau)}{d\tau}\cdot \cos t - \frac{\alpha}{\epsilon} B^3(\tau) \underbrace{\cos^3 t}_{\frac{1}{4}(3\cos t+ \cos 3t)}
\end{align}
What is dangerous? -- Terms proportional to $ \cos t $, which can blow off when $ t $ is large.  
Expand these terms to give 
\begin{align}
-2 \frac{d B(\tau)}{d\tau}- \frac{3}{4}\left(\frac{\alpha}{\epsilon} \right)B^3(\tau),
\end{align}
and put them equal to zero.
\begin{itemize}
\item This allows you to determine $ B(\tau) $. 
\item This ensures the removal of dangerous terms.
\end{itemize}

Interestingly, the zero-order approximation of this method has given a good approximation to the exact solution. The zero-order approximation gives
\begin{align}
x_0(t,\tau)=\frac{\sin t}{\sqrt{\frac{3}{4}\left(\frac{\alpha}{\epsilon} \right)\tau +1}}
\end{align}
Now, substitute $ \tau=\epsilon t $, we have
\begin{align}
x(t)\approx x_0 = \frac{\sin t}{\sqrt{\frac{3}{4}\alpha t +1}}.
\end{align}
